\documentclass[10pt,conference]{IEEEtran}

\usepackage{hyperref}

\usepackage{graphicx}
\usepackage{caption}
\usepackage{xcolor}
\usepackage{flushend}

\newcommand\todo[1]{\textcolor{red}{#1}}

\title{How to win a star on GitHub}

\author{
    \IEEEauthorblockN{Michel Kraaijeveld}
    \IEEEauthorblockA{Delft University of Technology\\
                    Delft, the Netherlands\\
                    Email: j.c.m.kraaijeveld@student.tudelft.nl\\}
    \and
    \IEEEauthorblockN{Tom den Braber}
    \IEEEauthorblockA{Delft University of Technology\\
                        Delft, the Netherlands\\
                    Email: t.d.denbraber@student.tudelft.nl}
}

\begin{document}

\maketitle
\IEEEdisplaynontitleabstractindextext

\begin{abstract}
	GitHub is a very successful open source software platform on which many projects are hosted.
	Since stars are included in the rankings on GitHub, they can be a good aspects to focus on when trying to stand out with a project.
	In this paper, a new dataset is presented based on GitHub data which consists of two thousand projects and multiple features.
	The dataset is created by using the GHTorrent project and making manual adjustments, such as adding domains to each of the projects.
	The paper also contains a model - that is used in combination with the created dataset - to predict the amount of stars for an open source software project.
	The model was applied by using multiple learning algorithms to find out whether prediction of stars is possible and to what extent.
	Experiments show that the model is able to predict the amount of stars with a high $R^2$ value, depending on the learning algorithm used.
	In combination with Random Forest and a log-scaled dataset, the model was able to obtain an $R^2$ value of 0.88.
	It shows that the predicted amount of stars is close to the actual number of stars.
\end{abstract}

\section{Introduction}
    More and more people make use of GitHub to host their Open Source Software (OSS). 
    GitHub is not only a place were people share their software, it has also become a `social network' for software developers because of it's social features. 
    One of those social features is `starring' a project. Users can show their appreciation for a project by giving it a  star. The amount of stars for a project thus gives an indication of how many people appreciate that project.
    
    However, there might be several factors that influence the amount of stars of a project. 
    One can think of the functionality that the project offers, but what about projects that offer more or less the same functionality but have a greatly different amount of stars?
    
    This paper will be looking at the stars given to OSS projects and whether certain factors can have influence on the amount of stars a project can possibly get, and if there are key points to focus on when aiming at having the most stars as possible.

    %\todo{
    %    \begin{itemize}
    %        \item What are stars?
    %        \item Why are stars cool?/what do they indicate?
    %        \item How is it possible to research them?
    %    \end{itemize}
    %}

%Uit 2011, maargoed toen al populair, dus tegenwoordig moeten zo nog wel meer hebben:
%https://github.com/blog/841-those-are-some-big-numbers
%https://github.com/blog/1204-notifications-stars

%http://www.iculture.nl/swift-populariteit-github-programmeertalen/
%https://www.githubarchive.org


%\section{Background}
To answer the questions specified earlier, a dataset which contains a lot of data concerning the projects on GitHub is needed. 
This dataset is available: GHTorrent \cite{gousios-2013} is a dataset which is constantly mining the GitHub application for more data on projects, and contains all the data that we need for this research.

\todo{More background info}

\section{Problem Description}
    Although the amount of stars for a project are far from the only indicator of a project's success, they give an indication of how many people actually like the project.
    For developers, it might be important to know how they can gain the appreciation of other GitHub users: the more people that use or collaborate to a project, the faster it will evolve.

    We will now further concretize the subject of this research. 
    The main research question is as follows: `Can we predict the amount of stars a project on GitHub will get when looking at its characteristics?'
    This high-level question can be split up into two more concrete questions:
    \begin{itemize}
        \item Which factors influence the popularity of a project?
        \item To what extent can we accurately predict the amount of stars on a project?
    \end{itemize}
    
    %\todo{waarom onderzoeken hierin, waarom is het nuttig, wat hebben mensen eraan}\\
    
    To answer the questions specified earlier, a dataset which contains a lot of data concerning the projects on GitHub is needed. 
    This dataset is available: GHTorrent \cite{gousios-2013} is a dataset which is constantly mining the GitHub application for more data on projects, and contains all the data that we need for this research.

    %\todo{More background info}

    However, since it also contains data that is not useful for this particular research, we chose to create our own dataset based on GHTorrent.
    This dataset contains 2000 open source projects with several features defined for each of them and is freely available for anyone to use.
    It can \todo{(not yet)} be found on https://github.com/IN4334-group-1/msr.


    %\todo{
    %\begin{itemize}
    %    \item Motivation
    %    \item Our research questions
    %    \item a short introduction into the dataset of Gousios
    %\end{itemize}
    %}


\section{Methodology}
    Now that the research questions have been discussed, the approach for answering those questions can be explained.
    We will start by detailing our sampling method. 
    Thereafter, we introduce the features which we think do have influence on the amount of stars that a repository has.
    Lastly, we discuss the methods for actually constructing a learning model that we can use to predict the amount of stars for a given project.
    \subsection{Sampling Method}
        When we did a first exploration of our data, we found that the distribution of stars per project is extremely right skewed: there are a lot of projects with one or zero stars, while there are much less projects which have a lot of stars, as can be seen in \todo{ADD FIGURE/TABLE}.
        Because of this skewness, the decision was made to use so-called stratified sampling.
        With stratified sampling, we first have to divide the population into homegenous strata. 
        Thereafter, we sample an equal amount of projects from each of these strata; 
        all these samples together form the sample that is being used in the rest of the project.
        
        Each stratum has as characteristic that the amount of stars of the project in that stratum fall within a certain range. 
        The ranges that will be used in this project are: [0, 10], (10, 100], (100, 1000], (1000, +].
        Because the dataset is quite large, we can also create a quite large sample:         
        out of each stratum, we will randomly select 500 projects for each stratum. The total size of our sample is thus 2000.
        
        Further, we split up our sample in a training set and a testing set. Out of our sample, we randomly pick 1000 projects which will form our training set. The remainder of the sample will be used as a testing set.
        
        Although we mentioned that the selecting of projects is completely random, it actually is not. In order to reduce bias in our sample, we decided to filter out any project by Google, Microsoft or Apple. 
        The way in which they use GitHub is different from most other projects, but they do have a lot of stars, partly due to brand awareness.
        

        
        
    
        \todo{
            \begin{itemize}
                \item Stratified sampling because of skewness
                \item Discuss strata intervals
                \item Discuss sample size
                \item Discuss items that are excluded from the sample (google/microsoft projects)
                \item Discuss training/testing split
            \end{itemize}
        }
    
    \subsection{Features}
        \todo{
            \begin{itemize}
                \item Discuss high level features
                \item Specify high level features into actual metrics
                \item Discuss adding of the `domain' to filter popular domain areas
            \end{itemize}
        }
    
    
    \subsection{Finding relations}
          \todo{
            \begin{itemize}
                \item Discuss machine learning algorithms (Multiple Linear Regression, Stepwise regression)
            \end{itemize}
        }


\section{Evaluation}
To make sure the created dataset contains the data we expect, there have been numerous validations on it.
Whenever the data of one of the features was retrieved, we manually checked some of the entries.
This was done by picking ten random samples from the data and looking them up on GitHub to see if the data we have found corresponds with the actual data.
Another point that we checked on, were the outliers in the data.
Whenever a feature contained entries that were much higher or lower than the other ones, we also manually checked them on GitHub to make sure the outliers are correct and not some anomalies in our data.
Some examples of this included a wiki with a single contributor (a bot) and over 100,000 commits and read-only repositories without contributors. 

\subsection{Threats to validity}
When creating the dataset, some issues arose which will be further explained in this section.
Also the steps taken to mitagate the threats, where possible, are described. \\

\subsubsection{Domains}
The projects in the dataset have been classified into different domains.
Since this has been done manually, it can occur that not all projects are classified in the right domains as it introduced subjectiveness.
Some projects also span multiple domains and that makes it hard to categorize it correctly.
Furthermore, the classification has been done only for the main categories.
The projects could have been classified into subcategories, but with a sample size of two thousand projects this would then result in small samples for each of the domains.\\

\subsubsection{Validations}
The data in the GHTorrent dataset is not fully up to date with the latest changes on GitHub.
This made it harder to validate the data that we retrieved, as it differs from the data we find when we look it up on GitHub.
An example of this would be the amount of stars a project has.
If GHTorrent contains data that is a couple of weeks old, the project might have gained or lost some stars in the meantime.
To be able to validate GHTorrent data even though it was not identical to the data currently on GitHub, we introduced a margin for the amounts that we can check.
This means that if a certain user was listed to have $x$ followers in GHTorrent, the user should have a number of  followers in the range of $[x - 0.05 \times x , x + 0.05 \times x]$ on GitHub.
This approach was also applied to the other features, such as stars and commits.\\

\subsubsection{GitHub limitations}
A limitation on GitHub's side is that deletions are not contained in the events that GHTorrent uses.\cite{gousios-2013}
This can introduce inconsistent data, as projects could have been deleted without this data being in the GHTorrent dataset.\\

\subsubsection{Unknown values}
Not all information on GithHub is filled in or can be determined.
Examples of this are the country of the developer that is often left blank, or the main language of the project that cannot be determined.
These values are saved in our dataset as `unknown' and often comprise a large portion of the data.
Because of that, estimations based on for example countries tend to point towards the unknown value.


\section{Results}
    \subsection{Initial Data Exploration}
    \todo{Add as much graphical content as possible here, maybe some first `insights'}
    
    \subsection{Testing the model}
        \todo{For laterrrr}


\section{Related Work}
Previous work conducted by Chen et al.\cite{chen-2014} aimed at predicting the amount of forks for a given repository on GitHub.
Our approach was similar to their's, as they had divided their data into different strata in order to get a good sample space.
Furthermore, they also had multiple features to create a model to predict the amount of forks.
The results of their research indicated that they could accurately predict the amount of forks for repositories on GitHub and provided valuable insights on the potentials for predicting properties of a GitHub project.

Recent research by Blincoe et al.\cite{blincoe-2015} on popularity of GitHub users, reveals that GitHub users can be influenced by each other to join new projects.
They carried out this research in two ways: by conducting a survey to find out motivations for people to follow others on GitHub and by repository analysis to see the actual influences.
Their findings showed that the main reason for following a user on GitHub, is the benefit of having updates and keeping up-to-date with the activities the user participates in.
These activities also include joining or discussing new projects, which popular GitHub user attract their followers to.

\todo{3rd paper (bissyande) \\
-Used dataset of 100,000 repositories (20,000 useful?) from github (from the github api) \\
-Looked at reported issues in those project and whether it has influence on the success of a project\\
-Shows that there is a correlation between the amount of issues and the popularity of the project, where popularity is based on stars (high correlation) and forks (very high correlation) \\


    \todo{\begin{itemize}
        \item Discuss RQs, methodology, results of \cite{chen-2014}
        \item Discuss RQs, methodology, results of \cite{blincoe-2015}
        \item Discuss RQs, methodology, results of \cite{bissyande-2013}
    \end{itemize}}


\section{Future Work}
Something that could be related to the amount of stars, that is not considered in this paper, is the documentation of a project.
When a project is documented well, or not documented at all, this might have an influence on the appreciation users show for that project.
The reason this was not considered in this paper, is because it is hard to measure documentations throughout different projects as some have extensive documentation on their own website, while they are not using the GitHub features for documenting the projected.
Furthermore, at the moment of writing it is not possible to check if a project has created a readme file or a wiki page in their repository, as the GHTorrent dataset does not include the files itself. For future research it might therefore be interesting to investigate this feature and include it in a prediction model.

%check that the projects you sample have no relation whatsoever one to another (e.g., that they do not share developers

\section{Conclusion}
	\todo{Probably: it's hard to predict the amount of stars ;p}
	
	\todo{
	    \begin{itemize}
	        \item A brand new dataset with all kinds of nice features, freely available for everyone
	        \item The amount of stars is hard to predict. or somethign.
	    \end{itemize}

\newline

\bibliography{references}{}
\bibliographystyle{ieeetr}

\end{document}
