\section{Introduction}
    More and more people make use of GitHub to host their Open Source Software (OSS). 
    GitHub is not only a place were people share their software, it has also become a `social network' for software developers because of it's social features. 
    One of those social features is `starring' a project. Users can show their appreciation for a project by giving it a  star. The amount of stars for a project thus gives an indication of how many people appreciate that project.
    
    However, there might be several factors that influence the amount of stars of a project. 
    One can think of the functionality that the project offers, but what about projects that offer more or less the same functionality but have a greatly different amount of stars?
    
    This paper will be looking at the stars given to OSS projects and whether certain factors can have influence on the amount of stars a project can possibly get, and if there are key points to focus on when aiming at having the most stars as possible.

    %\todo{
    %    \begin{itemize}
    %        \item What are stars?
    %        \item Why are stars cool?/what do they indicate?
    %        \item How is it possible to research them?
    %    \end{itemize}
    %}

%Uit 2011, maargoed toen al populair, dus tegenwoordig moeten zo nog wel meer hebben:
%https://github.com/blog/841-those-are-some-big-numbers
%https://github.com/blog/1204-notifications-stars

%http://www.iculture.nl/swift-populariteit-github-programmeertalen/
%https://www.githubarchive.org
