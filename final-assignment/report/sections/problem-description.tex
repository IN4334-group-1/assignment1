\section{Problem Description}
    Although the amount of stars for a project are far from the only indicator of a project's success, they give an indication of how many people actually like the project.
    For developers, it might be important to know how they can gain the appreciation of other GitHub users: the more people that use or collaborate to a project, the faster it will evolve.

    We will now further concretize the subject of this research. 
    The main research question is as follows: `Can we predict the amount of stars a project on GitHub will get when looking at its characteristics?'
    This high-level question can be split up into two more concrete questions:
    \begin{itemize}
        \item Which factors influence the popularity of a project?
        \item To what extent can we accurately predict the amount of stars on a project?
    \end{itemize}
    
    %\todo{waarom onderzoeken hierin, waarom is het nuttig, wat hebben mensen eraan}\\
    
    To answer the questions specified earlier, a dataset which contains a lot of data concerning the projects on GitHub is needed. 
    This dataset is available: GHTorrent \cite{gousios-2013} is a dataset which is constantly mining the GitHub application for more data on projects, and contains all the data that we need for this research.

    %\todo{More background info}

    However, since it also contains data that is not useful for this particular research, we chose to create our own dataset based on GHTorrent.
    This dataset contains 2000 open source projects with several features defined for each of them and is freely available for anyone to use.
    It can \todo{(not yet)} be found on https://github.com/IN4334-group-1/msr.


    %\todo{
    %\begin{itemize}
    %    \item Motivation
    %    \item Our research questions
    %    \item a short introduction into the dataset of Gousios
    %\end{itemize}
    %}
