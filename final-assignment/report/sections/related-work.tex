\section{Related Work}
Previous work conducted by Chen et al.\cite{chen-2014} aimed at predicting the amount of forks for a given repository on GitHub.
Our approach was similar to their's, as they had divided their data into different strata in order to get a good sample space.
Furthermore, they also had multiple features to create a model to predict the amount of forks.
The results of their research indicated that they could accurately predict the amount of forks for repositories on GitHub and provided valuable insights on the potentials for predicting properties of a GitHub project.

Recent research by Blincoe et al.\cite{blincoe-2015} on popularity of GitHub users, reveals that GitHub users can be influenced by each other to join new projects.
They carried out this research in two ways: by conducting a survey to find out motivations for people to follow others on GitHub and by repository analysis to see the actual influences.
Their findings showed that the main reason for following a user on GitHub, is the benefit of having updates and keeping up-to-date with the activities the user participates in.
These activities also include joining or discussing new projects, which popular GitHub user attract their followers to.

\todo{3rd paper (bissyande) \\
-Used dataset of 100,000 repositories (20,000 useful?) from github (from the github api) \\
-Looked at reported issues in those project and whether it has influence on the success of a project\\
-Shows that there is a correlation between the amount of issues and the popularity of the project, where popularity is based on stars (high correlation) and forks (very high correlation)} \\


    \todo{\begin{itemize}
        \item Discuss RQs, methodology, results of \cite{chen-2014}
        \item Discuss RQs, methodology, results of \cite{blincoe-2015}
        \item Discuss RQs, methodology, results of \cite{bissyande-2013}
    \end{itemize}}
