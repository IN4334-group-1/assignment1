\section{Related Work}
Previous work conducted by Chen et al. \cite{chen-2014} aimed at predicting the amount of forks for a given repository on GitHub.
Our approach was similar to their's, as they had divided their data into different strata in order to get a good sample space.
Furthermore, they also had multiple features to create a model to predict the amount of forks.
The results of their research indicated that they could accurately predict the amount of forks for repositories on GitHub and provided valuable insights on the potentials for predicting properties of a GitHub project.

Recent research by Blincoe et al. \cite{blincoe-2015} on popularity of GitHub users, reveals that GitHub users can be influenced by each other to join new projects.
They carried out this research in two ways: by conducting a survey to find out motivations for people to follow others on GitHub and by repository analysis to see the actual influences.
Their findings showed that the main reason for following a user on GitHub, is the benefit of having updates and keeping up-to-date with the activities the user participates in.
These activities also include joining or discussing new projects, which popular GitHub user attract their followers to.

Research \cite{bissyande-2013} into reported issues for projects on GitHub also provided useful insights.
They looked at 100.000 GitHub repositories to see which factors influence the amount of issues that are filed for a project.
Their results showed that there is a correlation between the amount of issues and the popularity of the project, where they based popularity on stars - which resulted in a high correlation - or forks - which resulted in a very high correlation.

\todo{Langslopen of de derde paper wel echt related work is.. \\
en of methodology van de papers uitgebreider moet dan nu het geval is}
